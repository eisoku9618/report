\documentclass[a4paper,xelatex,ja=standard,twocolumn]{bxjsarticle}

%% \XeLaTeXロゴ
\usepackage{metalogo}
%% subfigureのモダンな書き方のため
\usepackage[hang,small,bf]{caption}
\usepackage[subrefformat=parens]{subcaption}
\renewcommand{\subfigureautorefname}{\figureautorefname}
%% 箇条書き
\usepackage{enumerate}
%% bib
\usepackage[square,comma,numbers,sort&compress]{natbib}
%% ソースコードを貼り付けるため
\usepackage{listings, color}
%% 必須っぽい
\usepackage{amsmath, amssymb}
%% レイアウトのため
\usepackage{geometry, layout}
\geometry{top=2cm, bottom=2cm, left=1.5cm, right=1.5cm, includefoot}
%% リンクを貼るため
%% \usepackage[pdfborder={0 0 0}]{hyperref}
\usepackage[pdfborder={0 0 0}, colorlinks=true, pdfencoding=auto]{hyperref}
\newcommand*{\fullref}[1]{\hyperref[{#1}]{\autoref*{#1} \nameref*{#1}}}
\hypersetup{bookmarks=true,
        bookmarksnumbered=true,
        hidelinks,
        setpagesize=false,
        pdfauthor={EisokuKuroiwa}}

\author{Eisoku Kuroiwa}


\title{README}

\begin{document}
\maketitle

latexrun\footnote{\url{https://github.com/aclements/latexrun}}を使っている.
latexmkも良かったけど,latexrunは{\LaTeX}の出力が見やすいように整形されている点が便利なので使う.

git submodule でlatexrunを.latexrunにcloneして使っている.

自分用gitディレクトリにcdしてから,

git subtree add --squash https://github.com/eisoku9618/report.git master -P report

とすると良い.

\section{依存関係}
\begin{enumerate}
  \item {\TeX}Live 2010以降
    \begin{enumerate}
      \item tex --version で確認可能
    \end{enumerate}
  \item bxjsarticle最新版
    \begin{enumerate}
      \item mkdir -p ~/texmf/tex/latex
      \item cd ~/texmf/tex/latex
      \item git clone https://github.com/zr-tex8r/BXjscls.git
    \end{enumerate}
\end{enumerate}

\section{使い方}
\subsection{ビルド方法}
\begin{enumerate}
  \item make XXX.tex でXXX.texをビルドできる
  \item make XXX/YYY.tex でサブディレクトリのものもビルドできる
  \item make でサブディレクトリ含めて全部ビルドする
\end{enumerate}
\subsection{clean方法}
\begin{enumerate}
  \item make clean TARGET=./ で現在のディレクトリをcleanできる
  \item make clean TARGET=./XXX でディレクトリXXX以下をcleanできる
  \item make distclean でサブディレクトリ含めて全部をcleanできる
\end{enumerate}

\section{ディレクトリ構成}
{\TeX}ファイルを現在のディレクトリに置いても,サブディレクトリを掘っても使えるようにした.
今は使っていないが,スタイルファイルなど用のディレクトリを掘ったときようにTEXINPUTS="styles:"をつけている.

preamble.texとMakefileは全てに影響するファイル.

\section{できていないこと}
\begin{enumerate}
  \item bibliographystyle{junsrt}としたらエラーが出るけど良く分からない
  \item PDFを開いた時に左側に目次的なのが出るやつがhyperrefでできると思っているけど,出来ない
  \item usepackga{fancyhdr}でヘッダー・フッターをいじれるらしいが,奇数ページだけ何故かうまく行かない.
\end{enumerate}

\end{document}
