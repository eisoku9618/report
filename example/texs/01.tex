\section{{\TeX}で日本語の文章を書くときに気をつけること} \label{sec:no1}

処理系やドキュメントクラスなど色々あるが,確認事項としては,

\begin{enumerate}
 \item 日本語用ドキュメントクラスを使っているか
 \item 日本語とEnglishの間に「四分アキ」はあるか
 \item 日本語フォントは埋め込まれているか
\end{enumerate}

くらいがあるっぽい.
また,{\TeX}Liveは2010以降でないと日本語がutf-8で扱えないので,
Ubuntu14.04くらいからが良い.

\subsection{日本語用ドキュメントクラス}

まず,使用するドキュメントクラスが日本語用である必要がある.
理由は,日本語組版固有のスタイルがあるからで,それはそんなものかなという気がする.

最初にjarticleが作られたが,
規格に合っていなかったみたいでjsarticleが作られて,
jsarticleがplatexに強く依存していて他の処理系で使えなかったのでbxjsarticleが作られた,という流れ.
この文章はbxjsarticleを使っている.
なぜなら,platexやuplatexではなくて,xelatexを使いたかったから.
その理由は,latexrunを使いたかったから.
latexrunを使いたかったのは,出力を見やすくしてくれるから.

\subsection{四分アキ}

「CやC++では」,と書いたときに,日本語と英語との境目にスペースを入れるのが美しいらしく,そうなっているかどうか.
ちなみに,2016/02/15,Ubuntu14.04,bxjsarticleはgithubの最新の段階では,pdflatexを使うと四分アキが出来ず,xelatex / lualatexは四分アキが出来ていたが,lualatexよりもxelatexの方がビルドが早かったので,xelatexを使うことにした.

\subsection{フォントの埋め込み}
platexとdvipdfmxを組み合わせていた時は,dvipdfmxの-fオプションでフォントマップを指定しないと埋め込まれない,ということがあったが,pdflatexとかxelatexとかは何もしなくても埋め込まれる.

pdffonts xxx.pdfで確認可能で,embの欄がyesとなっていればいい.
