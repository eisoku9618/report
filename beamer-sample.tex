\documentclass[unicode,aspectratio=169]{beamer}
\usepackage{zxjatype}
\usepackage[ipaex]{zxjafont}
\usepackage{metalogo}

\usetheme{Berkeley}
\setbeamertemplate{footline}[frame number] %ページ番号

\begin{document}

\title{Beamer Sample}
\subtitle{サブタイトルはここに}
\author{eisoku9618}
\institute{所属はここに}
\date{\today}

\frame{\titlepage}

\frame{\frametitle{目次は自動生成です}\tableofcontents}


\section{section 1}
\frame{\frametitle{ここに文字を入れるとタイトルになる}
  はろーわーるど\\
  このサンプルは以下から拝借\\
  \url{http://www2.informatik.uni-freiburg.de/~frank/ENG/beamer/example/Beamer-class-example1.tex}
}
\subsection{Subsection no.1.1}
\frame{
  タイトル無しも可能
}


\section{section 2}
\subsection{list I}
\frame{\frametitle{番号なしリスト}
  \begin{itemize}
  \item これは \XeLaTeX を使っています.
  \item ので,日本語が入力できます.
  \item \XeLaTeX は \XeTeX 上で動く \LaTeX らしいです.
  \item テーマは Berkeley を使っています.左側がおしゃれなので.
  \end{itemize}
}

\frame{\frametitle{アニメーション的なことが pause で出来ます}
  \begin{itemize}
  \item まずこれ
    \pause
  \item 次にこれ
    \pause
  \item そして
    \pause
  \item 最後
  \end{itemize}
}

\subsection{list II}
\frame{\frametitle{番号付きリスト}
  \begin{enumerate}
  \item あ
  \item い
  \item う
  \end{enumerate}
}
\frame{\frametitle{こちらもpauseができます}
  \begin{enumerate}
  \item あ
  \item い
    \pause
  \item う
  \end{enumerate}
}

\section{section 3}
\subsection{Tables}
\frame{\frametitle{Tables}
  \begin{tabular}{|c|c|c|}
    \hline
    \textbf{Date} & \textbf{Instructor} & \textbf{Title} \\
    \hline
    WS 04/05 & Sascha Frank & First steps with  \LaTeX  \\
    \hline
    SS 05 & Sascha Frank & \LaTeX \ Course serial \\
    \hline
\end{tabular}}


\frame{\frametitle{Tables with pause}
  \begin{tabular}{c c c}
    A & B & C \\
    \pause
    1 & 2 & 3 \\
    \pause
    A & B & C \\
\end{tabular} }


\section{section 4}
\subsection{blocs}
\frame{\frametitle{blocs}

  \begin{block}{title of the bloc}
    bloc text
  \end{block}

  \begin{exampleblock}{title of the bloc}
    bloc text
  \end{exampleblock}


  \begin{alertblock}{title of the bloc}
    bloc text
  \end{alertblock}
}
\end{document}
